%%%%%%%%%%%%%%%%%%%%%%%%%%%%%%%%%%%%%%%%%%%%%%%%%%%%%%%%%%%%
% Pedro Brandão's trial to get a template for thesis for students
% Used the upthesis from Fernando Silva (see upthesis).
% See also the packages file.
% 2014/07/07 First draft
% 2014/07/21 
%  pbrandao: added the list of listings (it should produce portuguese name if
%           babel is set to portuguese, see packages.tex). Changed usepackage of babel
%           to be before input packages.tex to allow test

%
%%%%%%%%%%%%%%%%%%%%%%%%%%%%%%%%%%%%%%%%%%%%%%%%%%%%%%%%%%%%


% makes all pages the height of the text on that page. No extra vertical space is added.
\raggedbottom 
% setting it to report will remove the blank pages before each chapter
\documentclass[11pt,a4paper,twoside]{book}


%%%%%%%%%%%%%%%%%%%%%%%%%%%%%%%%%%%%%%%%%%%%%%%%%%%%%%%%%%%%
%%%   Packages that need to be configured for the thesis
%%%%%%%%%%%%%%%%%%%%%%%%%%%%%%%%%%%%%%%%%%%%%%%%%%%%%%%%%%%%

% Language settings
% use UKenglish for UK or leave blank for US English
% it will also change the names for some of the chapters (list of tables, figures, content,
\usepackage[UKenglish]{babel}
%\usepackage[portuguese]{babel}

%%%%%%%%%%%%%%%%%%%%%%%%%%%%%%%%%%%%%%%%%%%%%%%%%%%%%%%%%%%%
%%%   Packages uses language definitions
% see file below for more packages and settings
%%%%%%%%%%%%%%%%%%%%%%%%%%%%%%%%%%%%%%%%%%%%%%%%%%%%%%%%%%%%
\usepackage{upthesis}

\usepackage[
%backref={section},
%pagebackref, % for getting references to the page where the citation is (in the biblio)
pdfpagelabels=false
]{hyperref}
\hypersetup{pdftitle={Titulo da tese}, %nao suporta acentos
   pdfkeywords ={palavras chave},
   pdfsubject = {assunto},
	bookmarksnumbered=true,
   pdfauthor ={Autor}, % see other options on manual (can be page) needs empty line on bibitem
   plainpages=false, 
   pdfborder={0 0 0},
   colorlinks,%colorlinks=false,
   breaklinks=true,
	%linktocpage= false, make page number, not text, be link on TOC, LOF and LOT 
	%hyperindex=true 	% Makes the page numbers of index entries into hyperlinks. Relays on unique page anchors (pageanchor) 
% see for colors http://mirror.ctan.org/macros/latex/contrib/xcolor/xcolor.pdf
   linkcolor=	Sepia, %MidnightBlue,% BlueViolet,%Sepia, % Color for normal internal links.
   %anchorcolor=black,% Color for anchor text.
   citecolor=RedViolet,% Color for bibliographical citations in text.
   %filecolor=cyan% Color for URLs which open local files.
   %menucolor=red% Color for Acrobat menu items.
   %runcolor=filecolor% Color for run links (launch annotations).
   urlcolor=NavyBlue% Color for linked URLs.
}
%use the same style for \url as the text
% from http://en.wikibooks.org/wiki/LaTeX/Hyperlinks#Customization
\urlstyle{same}


\begin{document}

\title{Motif based community discovery}
\submitionplace{Tese submetida à Faculdade de Ciências da \\
  Universidade do Porto para obtenção do grau de Mestre \\ 
  em Ciência de Computadores}
\author{Rui Miguel Capela Fonseca}
\department{Departamento de Ciência de Computadores \\ Faculdade de
  Ciências da Universidade do Porto}
\submitdate{Junho 2018}

%\includepdf[pages=1]{FrontPage-MSc.pdf}
%\cleardoublepage
%\includepdf[pages=2]{FrontPage-MSc.pdf}
%\cleardoublepage

\beforepreface%

\prefacesection{Abstract}

1) Pequena introdução ao tema
2) Motivação para o desenvolvimento do trabalho
3) Definição dos objectivos propostos para o trabalho
4) Revisão do estado da arte
5) Estado corrente do trabalho e plano para o desenvolvimento seguinte.
6) Referências

* Networks, Complex networks: what they are, what they represent in the real world, what are they used to, what is their value

* Communities: What is a community, what do they represent, what is the value of finding a community, how can we find a community, why is it so hard

*Motifs: what is a motif, what do they represent, how to find motifs in networks, how can motifs give value to community discovery


\prefacesection{Resumo}
Há muito, muito tempo

See the Abstract.


\prefacesection{Agradecimentos}

Obrigado a todos, obrigado \ldots


\dedicationpage{Dedico a \ldots}

% end of thesis preamble
\afterpreface%

%% main tex here
%% By putting the chapter names here, one can just comment the content in the chapters 
%% and produce a pdf with the correct chapter number.
%% If you want further configurability you can use subfiles package
%% https://www.ctan.org/pkg/subfiles

%\chapter{Introdução}\label{chap:intro}
\chapter{Introdução}\label{chap:intro}
This should be a summary of what comes in the next chapters. Here you explain in two to five
pages: \\(1) the context of your work highlighting and defining the problem you need to solve, \\
(2) what you want to do (objectives) and (2) why you want to do it (motivation), \\
(3) how you want to
achieve your objectives (methodology), always supporting your text on the available literature, \\
(4)
contributions (Results that confirm that you achieved your objectives) and \\
(5) organization of the
chapters that come next.\\

performance, aplicaçoes, metologias( novas formas)

desenvolver uma ferramente em cima do gephi

tentar reproduzir o motif based cominities algorithms, path e cycle modularity

networks base, as simples do paper e por exemplo as do karate

o codigo do ribeiro está em c++, é possivel usar com outra linguagem só a ler o output mas não fica tao facil mudar a implementaçao se for preciso ou der jeito

* Networks, Complex networks: what they are, what they represent in the real world, what are they used to, what is their value

* Communities: What is a community, what do they represent, what is the value of finding a community, how can we find a community, why is it so hard

*Motifs: what is a motif, what do they represent, how to find motifs in networks, how can motifs give value to community discovery






%\chapter{Background}\label{chap:back}
\chapter{Background}\label{chap:back}

This chapter has the ``Instructions for preparing and writing M.Sc. Dissertations'' (Research-oriented work), 
Version 1.0, January, 7$^{th}$, 2015 from prof. Inês Dutra. It was originally written in English so it was kept as such. It was some additions from Pedro Brandão.

\section{Before Starting}
%%%%%%%%%%%%%%%%%%%%%%%%%%%%%%%%%%%%%%%%
Before starting your dissertation, you need to \textbf{define} what is the subject you are going to work
with and perform a \textbf{thorough systematic} bibliography review of the theme you chose.
What is a systematic review? It is the one where you search the web or books for the subject, and
define rules for filtering papers in two sets “included” and “excluded” and explain why some papers
go to one set or the other.
In order to start the search, you need to prepare keywords related to your subject and prepare
queries to be used in Google Scholar, Scopus, MesH etc. These search engines will return a number
of papers on the subject you are looking for.\textbf{You need to read at least the abstract and 
conclusions of every paper retrieved after your search}. Now, you filter out only the ones you
think are very closely related to your research work, and give a reason for choosing those papers.

\section{Bibliography}
%%%%%%%%%%%%%%%%%%%%%%%%%%%%%%%%%%%%%%%%
 Start organizing your bibliography file. Choose one of the standards available to start organizing
your references. Usually your department/faculty has clear rules about the standard to be used. If
you are formatting your text using \LaTeX, most references found during your bibliographic search
can be exported in BibTeX format.

\subsection{Bibliography Section}
%---------------------------------------
Your dissertation needs to have a Bibliography Section with a list of the cited works you have in
the text. In the process of writing your dissertation, make sure to properly refer the authors you are
basing your text on. For example, “Yaacoub et al.~\cite{yaacoub2012} discovered that\ldots”. In this sentence, Yaacoub is the
first author of one of the publications you list in the Bibliography Section of your dissertation and~\cite{yaacoub2012} is the link that connects this citation to the publication in the bibliographic list. If your
bibliographic entry has only one author, you cite only the author's surname. If the bibliographic
entry has two authors you cite the two authors' surnames (e.g., Clausen and Jacquet~\cite{Clausen2003}). If the
bibliographic entry has more than two authors, you can use the expression “et al.”, like in the
example shown before.

You should avoid as much as possible web site references. Only in some cases, illustration, showing trends, are they acceptable.

References should be used to back up claims made. Specially in the introduction section, sentences that stipulate something should be backed by references that assert that claim (e.g.: Android, in December 2016, was the most widely used mobile operating system~\cite{netMarketShareMobileOS}\footnote{An example where a web link to a recognized market analysis company would be valid. Note that the time which the report was seen is very relevant.}

\section{Inserts}
%%%%%%%%%%%%%%%%%%%%%%%%%%%%%%%%%%%%%%%%
Every picture, graph, diagram, algorithm etc needs to have a caption and a number and needs to
be cited and explained in the text. The mere existence of a picture, etc. does not exempt it of a description. 

\subsection{Copyrights and image usage}
%---------------------------------------
If you want to use any picture, graph, diagram etc available in one of the publications in your
dissertation, you need to make sure that you can use it (check the copyright rules). If the copyright
rules allow you to reproduce the picture (or others) in your text, you need to insert a reference to the
source (where the picture was taken from) in the caption. If you are allowed to use a picture, but
want to slightly modify it, you need to say in the caption: Adapted from [1] (where [1] is the
number of your reference in the Bibliographic Section).


\section{Chapters/Organization}
%%%%%%%%%%%%%%%%%%%%%%%%%%%%%%%%%%%%%%%%
\begin{description}
   \item[Chapter 1: Introduction:]
This should be a summary of what comes in the next chapters. Here you explain in two to five
pages: (1) the context of your work highlighting and defining the problem you need to solve, (2)
what you want to do (objectives) and (2) why you want to do it (motivation), (3) how you want to
achieve your objectives (methodology), always supporting your text on the available literature, (4)
contributions (Results that confirm that you achieved your objectives) and (5) organization of the
chapters that come next.
\item[Chapter 2: Basic Concepts:]
In this chapter you need to present the foundations of your work: theoretical aspects, background
material etc, all that is needed to understand the terminology and expressions used in the remaining
chapters.
\item[Chapter 3: Related Work:]
Here you need to discuss about other works in the literature that do something similar to what
you want to do. You need to cite and discuss the relevant papers you chose to include in your study
during your survey. Explain what others do, why it is not sufficient, and why you need to do what
you want to do. It is helpful to define some criteria to compare your work against others, and
build a table with main characteristics of other works contrasting to what you want to do. In other
words, in which aspects is your work different from others?
\item[Chapter 4: Your Work:] this chapter describes the contributions of the work done. If it is based on prior work (continuation of the project or using prior developed work), the should only describe what is the new work done. If references are needed, it should be clear what is the prior work and what is the new contribution.
\item[Chapter 5: Materials and Methods:] should have:
   \begin{itemize}
      \item Definition of Experiments (if any)
      \item Definition of Evaluation Metrics
   \end{itemize}
\item[Chapter 6: Results and Analysis]
\item[Chapter 7: Conclusions and Future work:] 
         this should restate the problem and iterate through the solution(s) analyzing the advantages and contributions. The limitations and unsolved problems should also be described.
         It should also describe the potentiality of new research/development that the work enables, the future work.
   \begin{itemize}
      \item Research Summary
      \item Main Findings
      \item Limitations
      \item Future Work
      \item Conclusion
   \end{itemize}
\end{description}
During writing, some of these chapters may collapse into just one.\textbf{Your work (chapters 4-7) should account for at least 50\% of your whole dissertation.}

\subsection{Contents of each chapter}
%---------------------------------------
You should start each chapter with a summary of its objective and contents, preferably relating to previous ones. At the end of the chapter provide a conclusion/summary of it, preferably connecting it to the next one.

%vim: set fo+=aw tw=80 spl=en_gb spell: syntax spell toplevel  :


%\chapter{Estado da Arte}\label{chap:stat}
\chapter{State of the Art}\label{chap:stat}


In this chapter I will present and discuss related work in the literature.

In the work Benson et al.~\cite{Benson2016} set out the objective to analyze and discover rich connectivity patterns in complex networks. They developed a framework that given a network and a motif finds a subset of nodes S by optimizing a metric called motif conductance [Should I cite this term?]. This defined metric generalizes the conductance metric in spectral graph theory, it accounts for the presence of motif instances inside the subset and avoids cutting motif occurrences. The optimization framework provably finds near-optimal clusters by extending the spectral graph clustering methodology. They apply eigenvalues and eigenvectors to a computed motif adjacency matrix which produces various nested sets Sr =\{[OMEGA SYMBOL]1, …, sr\} of increasing size in r. They  prove that the set Sr with the smallest motif-based conductance is a near-optimal higher-order cluster. This results in a computational efficient and easy to implement program with mathematical guarantees on the near-optimality of obtained clusters (at most a quadratic factor away from optimal).

Various applications are shown on real networks like the C. elegans neural network [SHOULD I CITE THIS?] and air traffic in Canada and the United States.

A restriction for this metric and framework is that it only creates a subset of the network S and its complement, so in practice we get two sets of nodes who also try to have the same number of motif instances in them. This means that we can not find more than two clusters for some motif. The authors are able to show results of more than two clusters by using analyzing various runs with different motifs and then joining the information obtained. 


Arenas et. al~\cite{Arenas2008} question the limitations that standard Newman-Girvan [SHOULD CITE?] modularity impose. The popular metric provides a values for some partition of the network that depends on how much the nodes of a community are linked to each other and how many connections exist to the outside communities. With this metric we are left with a problem of optimization which various works have tried to solve [SHOULD CITE REFERENCES?]. In modularity are only looking to the simplest connectivity pattern that we could - are these two nodes connected or not - and while this can be enough in some cases, higher order complex networks often have hidden structures that can be revealed with more complex connectivity patterns. So a new definition called motif modularity is proposed, an extension of the standard modularity, that can be applied to undirected, directed, weighted and unweighted networks. Some specializations of the definition are also shown described, such as cycle modularity with the triangle motif (three nodes connected to each other) and path modularity, defined by a linear motif of length l Ep(l) = \{(1, 2), (2, 3),... , (l, l + 1)\} 

Cycle modularity of size three is used to find highly connected clusters in much the same way standard modularity works. It is used in a synthetic network and the Zachary Karate Club Network [SHOULD CITE] to reveal areas of high connectivity, resulting in desired partitions.

With path modularity for a path of size three and the adding restriction that the first and third node belong to the same class, they are able to find a bipartite structure in networks that have it. They also extended it to find multipartite structure like the Southern Women Event Participation network [SHOULD CITE].

We still have a problem left to solve which is how to find the partition that maximizes this new definition - motif modularity - since it is infeasible to compute all different partitions for some network.


Milo et. al \cite{Milo2002} study the role of connectivity patterns in uncovering hidden structures in complex networks. They define "network motifs" as sequences or patterns of nodes and edges that connect them. Complex networks often contain an abnormal quantity of some of these motifs when compared to randomized networks, so we could deduce reasons why those patterns appear.

They found networks with these properties from areas like biochemistry, neurobiology, ecology, and engineering. Different networks with the same hidden structure - for example various electronic chips with digital fractional multipliers have the same type of motifs over represented. Even networks from different areas but similar functions, like the C. elegans neural network and electronic circuits with with forward logic chips have the same over represented motifs in common. 

This network motif definition and the fact that complex networks can express their inner structure by analyzing this motifs opens the door to new methodologies to describe networks.

Fortunato and Barthélemy~\cite{Fortunato2007} noted the influence of modularity [SHOULD CITE?] as a way to qualify community discovery and in clustering methods through optimization. They mathematically prove that this optimization may hide the real partition because having two communities of some size small enough is going to yield a worse modularity value than if those two communities where joined in a single one.

This brings some problems to the table for methods that rely on the optimization of modularity and where the real communities for a network have heterogeneous size. So it becomes necessary to perform tests in order to check for the existence possible biases and resolution limits in the results.

Arenas et. al ~\cite{Arenas2008ResolutionLimit} try to answer to the problem of resolution limit in the modularity optimization task. They come up with a variable that represents how much we want to prefer smaller communities. When it is zero, it represents the normal network and when we increase the variable we have access to more refined substructures. What is they want to find is the minimum value for that variable that maximizes the modularity. To do that, they use two heuristics for compute motif modularity: extremal optimization and tabu search. They show the results of applications to synthetic networks and how ranges values of the variable can create plateaus of stability, provable points of interest. They mention that all ranges where the modules stay stable are of importance, not only the most stable range.

So to solve a resolution limit that may exist one might apply the methods described [SHOULD CITE?] and analyze the points of interest. 

\chapter{Work Done}\label{chap:work}

How did I start?

I started with Ribeiro's code it was a good starting point to bootstrap my program, it contained the fundamental data structures that I will use and also the helper functions to manipulate and analyze them. I made sure I could read a network correctly from a file and the helper functions did the correct operation that I expected.

I added the kronecker function that is referenced in Arenas et. al \cite{}

%\chapter{Desenho da Arquitetura}\label{chap:syst}
\chapter{Desenho e Desenvolvimento}\label{chap:devel}

O referente ao ``Your work'' do capítulo~\ref{chap:back}.

No contexto deste template, este capítulo serve de exemplos de uso do \LaTeX e de algumas regras de tipografia. Note-se que as regras sobre a formatação estão disponíveis \href{https://sigarra.up.pt/fcup/pt/conteudos_geral.ver?pct_pag_id=1011511&pct_parametros=pv_unidade=97&pct_grupo=33673&pct_grupo=33670&pct_grupo=33675&pct_grupo=33683&pct_grupo=36711#36711}{Estrutura e Layout de teses}. Pode-se ver exemplos de tese (para a formatação e conteúdo) no repositório da UP \url{https://repositorio-aberto.up.pt/handle/10216/9535}.

\section{Exemplo de código}
%%%%%%%%%%%%%%%%%%%%%%%%%%%%%%%%%%%%%%%%

\begin{lstlisting}[numbers=none,language=java,caption={[CommandDaemonCallsItf]
   {CommandDaemon} callback interfaces},label=lis:commandDCallsItfs,float=htb]
public interface CallBackCmdMeasurements { // comment
	public abstract void newMeasure(MeasurementBasic measure, int reqId);
	public abstract void newMeasuresAggSimp(MeasurementBasic[] measuresAggSimp, 'A string');
}
\end{lstlisting}

É possível como referir o código, por exemplo o bloco de código~\ref{lis:commandDCallsItfs}.

\section{Acrónimos}
%%%%%%%%%%%%%%%%%%%%%%%%%%%%%%%%%%%%%%%%

Deve-se acrescentar os acrónimos no ficheiro \texttt{acros.tex} e ordená-los 
alfabeticamente nesse ficheiro.
Vamos usar o acrónimo \ac{TCP} que deve estar expandido, assim como no 
capítulo~\ref{chap:tests}. Os acrónimos devem aparecer expandidos em cada 
capítulo (o que está já configurado para esta dissertação).

Ao usar $\backslash$\texttt{acp} o \texttt{acronym} tentará colocar o plural 
(acrescentando um s). É possível no ficheiro \texttt{acros.tex} colocar qual o 
plural pretendido por exemplo \acp{USF}. Deve-se usar 
$\backslash$\texttt{newacroplural\{USF\}\{Unidades de Saúde Familiar\}}, 
definindo o novo plural. Pode-se opcionalmente definir um novo ``plural'' para a 
versão do acrónimo com 
$\backslash$\texttt{newacroplural\[USFes\]\{USF\}\{Unidades de Saúde Familiar\}} 
(acrónimo no plural incorreto; apenas para exemplo).

Podem usar $\backslash$\texttt{acs} para apenas mostrar o acrónimo, 
$\backslash$\texttt{acl} para mostrar a expansão. Ver mais na documentação do 
pacote \texttt{acronym}.

\section{Figuras}
%%%%%%%%%%%%%%%%%%%%%%%%%%%%%%%%%%%%%%%%

Podem ver a figura~\ref{fig:logoFCUP} muito bem. Notem que na lista de figuras não aparece tudo o que está na legenda, mas apenas o que está entre~[].

\begin{figure}[htb]
   \centering % center the figure
   \includegraphics[scale=.4]{pics/fc_logo}
   \caption[FCUP logo velho]{O logo da FCUP antigo}\label{fig:logoFCUP}
\end{figure}

As figuras devem usar o posicionamento [htb] para serem preferencialmente colocadas no local onde foram declaradas.

\subsection{Formato}
%---------------------------------------
    Devem colocar as figuras sempre que possível em pdf. Usem o programa (ou aplicação web) para exportar para pdf. Desse modo a qualidade das mesmas é sempre maior.

    No caso em que o pdf gerado ocupe uma página A4 inteira (por exemplo nas ferramentas do MS Office) podem usar o \texttt{\href{https://www.ctan.org/pkg/pdfcrop?lang=en}{pdfcrop}} no pdf gerado. Este comando está disponível na maior das distribuições de \LaTeX.


\section{Referências a bibliografia}
%%%%%%%%%%%%%%%%%%%%%%%%%%%%%%%%%%%%%%%%
Algumas referências para se ver a ordenação~\cite{yaacoub2012} (estão ordenadas pelo apelido do autor). Aqui outra ainda~\cite{etsitr102732}.

Temos aqui o Clausen~\cite{Clausen2003}. E para ver várias~\cite{yaacoub2012, etsitr102732, strunk2007elements}.

No ficheiro bib de referências deve-se colocar entre \{\} o que se quiser manter com maiúsculas no título. Exemplo: \texttt{title = \{Addressing the \{WLAN\} Problem\}} resulta em \texttt{Addressing the WLAN problem}, note-se o \textbf{p} de \emph{problem}. 

Para saber os campos do bibtex relevantes para cada tipo de entrada pode-se utilizar um editor de bibtex como o \href{http://www.jabref.org}{JabRef}.

Os meses indicados no ficheiro \texttt{bib} poderão aparecer em inglês, caso se pretenda deve-se colocar o texto em português explicitamente no ficheiro \texttt{bib}.


\section{Capa}
%%%%%%%%%%%%%%%%%%%%%%%%%%%%%%%%%%%%%%%%
Para a inclusão da formatação correta para a capa, pode-se criar um pdf usando o 
pptx disponível 
(\texttt{\href{https://sigarra.up.pt/fcup/pt/conteudos_service.conteudos_cont?pct_id=162960&pv_cod=19anaMgwFYkm}{CapaMSc\_PowerPoint.pptx}}, 
necessário fazer login no Sigarra) e depois incluí-lo com o comando seguinte 
(comentado no tex original).

\begin{lstlisting}[numbers=none,language=TeX,caption={[Capas tese] Incluir capas oficiais},label=lis:capasTese,float=htb]
\includepdf[pages=1]{FrontPage-MSc.pdf}
\cleardoublepage
\includepdf[pages=2]{FrontPage-MSc.pdf}
\cleardoublepage
\end{lstlisting}


\section{Tabelas}
%%%%%%%%%%%%%%%%%%%%%%%%%%%%%%%%%%%%%%%%
Exemplo de uma tabela mais complexa na tabela~\ref{tab:bsnvswsn}.

\begin{table}[htbp]
   \caption[BSN vs WSN]{BSN versus WSN
   (with input from  Latré  and Guang )}
\label{tab:bsnvswsn}
\centering
{
\footnotesize
\begin{tabularx}{0.98\textwidth}{|>{\columncolor{gray-cell}}c|X|X|}
   \hline
   \rowcolor{gray-cell} 
   &    \centering  \textbf{BSN} &  \centering \textbf{WSN}  \tabularnewline 
   \hline
   %%%% line
   \begin{sideways} \hspace{-11em} \textbf{Distribution} \end{sideways}
   &  
   \begin{asparaenum}[\bfseries i)]
      \item Existence of a \ac{BS};
      \item \ac{BS} collects, maintains and processes the data;
      \item Nodes will do minimal processing, sending all data to the \ac{BS};
      \item Centralized system where \ac{BS} controls all nodes;
      \item Node replacement is difficult in in-body sensor nodes;
      \item Smaller number of nodes;
      \item Nodes need to take biocompatibility, wearability into account.
   \end{asparaenum}
   & 
   \begin{asparaenum}[\bfseries i)]

      \item A \ac{BS} may or not exist or there may be several \acp{BS} (e.g. mobile nodes 
         collect info, clustering);
      \item As in \ac{BSN}, but also on-demand querying;
      \item Nodes will do processing, aggregation to alleviate communication or 
         correlate results;
      \item Distributed system, nodes decide cooperatively;
         %(form clusters, aggregate data, \etc);
      \item Node replacement is difficult due to location, scale, etc.;
      \item (usually) Wide areas covered by large number of nodes.
      \item Nodes may need to be environment friendly, indiscernible from surroundings.
     %    \vspace{-0.7em}
   \end{asparaenum}
   \tabularnewline \hline
   %%%% line
   \textbf{Comm.} 
   & 
   \begin{asparaenum}[\bfseries i)]
      \item One hop to \ac{BS};
      \item Close range but attenuated by body;
      \item Data rates heterogeneous.
    %\vspace{-0.7em}
   \end{asparaenum}
   & 
   \begin{asparaenum}[\bfseries i)]
      \item Multi hop through network of sensor nodes;
      \item Long(er) range;
      \item Data rates homogeneous.
    %\vspace{-0.7em}
   \end{asparaenum}
   \tabularnewline \hline
   %%%% line
    \textbf{Data} 
   & 
   This is some text on this cell. The multirow package does not know the height of the cell and can not center the cell to the right. This is because of the X from tabularx.
   & 
   \multirow{ 2}{*}{This is on two rows of the table}

 \tabularnewline \cline{1-2}\noalign{\vskip.3pt}% using the noaling vskip to show the line
   %%%% line
    \textbf{Energy} 
   & 
   Some more text just to show something
   %      \vspace{-0.7em}
   & 
    \tabularnewline \hline
\end{tabularx}
}
\end{table}

Uma tabela de lado exemplificada em~\ref{tab:sensorExample}.
\begin{sidewaystable}
   \centering
   \begin{threeparttable}
      \caption{Sensor examples}\label{tab:sensorExample}
      {
      \footnotesize
      %\begin{tabularx}{0.99\textheight}{|m{9.5em}|m{7.5em}|>{\columncolor{gray-cell}}m{12em}|m{12em}|c|c|m{5em}|}
      \begin{tabular}{|>{\centering}m{9.5em}|>{\centering}m{6.5em}|>{\columncolor{gray-cell}\centering}m{14em}|m{15em}|c|c|>{\centering}m{6.5em}|}
         \hline
         \rowcolor{gray-cell} 
          \textbf{Device} &  \textbf{Availability}&  \textbf{Sensed} & \centering \textbf{Technology} & 
         \textbf {Frequency} & \textbf{Data Rate} \tnote{$\mathbf{\propto}$} &     \textbf{Energy} \tnote{$\mathbf{\Diamond}$}
         \tabularnewline \hline

         %%%% line
         \multirow{6}{*}{\vspace{-3em}BioHarness BT} &
         \multirow{6}{*}{\vspace{-3em}commercial}
         & \acs{HR} & detection of \emph{QRS} complex in \acs{ECG}  &
         1~Hz & 8~bps &
         \multirow{6}{5.5em}{\centering 21~h transmitting}
         \tabularnewline \cline{3-6}

         %%%% line
          &
         & Breathing rate & conductive elastic measurement of thorax excursion & 1~Hz & 7~bps &
          
         \tabularnewline \cline{3-6}

         %%%% line
          &
         & 3D Accelerometer & variability of a weight reference  & 50~Hz \tnote{$\Join$}& 500~bps \tnote{$\Join$} &
          
         \tabularnewline \cline{3-6}

         %%%% line
          &
         & \acs{ECG} & potential difference across electrodes in body &  250~Hz &
         2500~bps &
          
         \tabularnewline \cline{3-6}

         %%%% line
          &
         & Gyroscope & angular momentum  & 1~Hz & 9~bps&
          
          \tabularnewline \cline{3-6}

          &
         & Skin temperature & thermistor  & 1~Hz & 9~bps&
          
         \tabularnewline \hline
      %%%%%%%%%%%%%%%%%%%%%%%%%%%%%%%%
       %%%% line
        Actigraph GT3X+~\cite{actiGSpecs} &
         commercial
         & 3D Accelerometer & variability of a weight reference  & 30-100~Hz \tnote{$\Join$} &
         360-1200~bps \tnote{$\Join$} &
         31 days
         \tabularnewline \hline

       %%%%%%%%%%%%%%%%%%%%%%%%%%%%%%%%
       %%%% line
        Shimmer Research GSR Sensor &
         commercial
         & Galvanic skin response & measure skin conductivity & up to 15.9~Hz &
         191~bps &
         60 $\mu$A  
         \tabularnewline \hline
       %%%%%%%%%%%%%%%%%%%%%%%%%%%%%%%%
       %%%% line
         \multirow{2}{*}{Nonin Onyx II~\cite{noninOnyxSpecs}} &
         \multirow{2}{*}{commercial}
         & $\mathrm{O_2}$ saturation (full waveform) & \multirow{2}{15em}{measure light absorption by blood haemoglobin}  &
         75~Hz & 1200~bps &
         \multirow{2}{6.5em}{\centering 2$\times$1.5~V AAA (600 tests)}
         \tabularnewline \cline{3-3} \cline{5-6}

         %%%% line
          &
         & $\mathrm{O_2}$  saturation (display format)  &   &
         1~Hz & 8~bps &
         
         \tabularnewline \hline

         %%%% line
         Medtronic iPro CGM & commercial
         & continuous glucose meter & electrochemical detection of glucose through its reaction with glucose oxidase
         & 0.1~Hz \tnote{$\lhd$} & 1~bps \tnote{$\rhd$}&
         up to 72~h
         \tabularnewline \hline

         %%%% line
         Brain sensor by Nurmikko et al. & in research \tnote{$\ddag$} &  brain activity &  microelectrode arrays detect neuron firing & \multicolumn{2}{c|}{40~k samples/sec$\times$16 channels \tnote{$\aleph$}} &
        12~mW 
         \tabularnewline \hline

         %%%% line
         Molecular biomarkers by Ling et al. & research prototype \tnote{$\star$} &  serum cardiac troponin I, creatinine kinase, myoglobin &  magnetic properties of sensors vary according to presence of biomarker &
         \multicolumn{3}{c|}{N/A} 
         \tabularnewline \hline

          %%%% line
          Electrochemical dopamine sensor by Chan et al.\tnote{$\otimes$} & research  &  dopamine &  Interdigitated micro electrodes measured electrochemical reaction& 50~Hz & -- & 10~pA
         \tabularnewline \hline

      %\end{tabularx}
      \end{tabular}
      \begin{tablenotes}[online]
      \item[$\propto $] data rates are based on the frequency and the accuracy in bits
         stated in the references; they do not include time stamps or message headers.
      \item[$\Diamond$] values taken from the references as available.
      \item[$\Join$] for each of the 3 axis.
      \item [$\lhd$] sensor data is collected every 10~s by collector.
      \item [$\rhd$] our assumption of 9 bits per measure (up to 512~mg/dL (\acs{USA} glucose units)).
      \item[$\ddag$] prototypes and clinical trials exist for devices with 16 channels.
      \item[$\aleph$] more channels are needed (e.g. for decoding arm joint angles); 100 channels arrays are being developed.
      \item[$\star$] tested on mice; measurement using \ac{MRI}, but test values were extracted from explanted sensors.
      \item [$\otimes$] the developed sensor was not made of nano tubes, which according to the authors, led to poor sensitivity.
      \end{tablenotes}
      }
   \end{threeparttable}
\end{sidewaystable}


\section{Unidades}
%%%%%%%%%%%%%%%%%%%%%%%%%%%%%%%%%%%%%%%%
Deve-se usar o sistema ISO de unidades e respeitar a capitalização das unidades de acordo com a norma. Ver \emph{Quantities and units} em \href{http://www.cl.cam.ac.uk/~mgk25/publ-tips/#typography}{Typographic conventions} (podem e devem dar uma olhada nas outras partes da página).

\section{Sugestões na escrita e uso do tex}
%%%%%%%%%%%%%%%%%%%%%%%%%%%%%%%%%%%%%%%%
\begin{itemize}
  \item usar o \textasciitilde{} para ligar as referências, evita a possível mudança de página: ex.: como o Brandão refere\textasciitilde\textbackslash{}cite\{bran99\}
     \begin{itemize}
        \item também se deve seguir a mesma regra para \textbackslash{}ref, unidades de dados, nº de standard (ex.: IEEE\textasciitilde802.11).
     \end{itemize}

\end{itemize}
\section{Sugestões na escrita}
\begin{itemize}
  \item em inglês não usar a \emph{short form}. Em textos formais deve-se manter a \emph{long form}. Incorreto: ``don't use''. \textbf{Correto:} ``do not use''
  \item em inglês (e também português) não usar a forma reflexiva ou indireta, preferir sempre a forma direta (que é mais assertiva e torna as frases menos complexas).
     \begin{itemize}
        \item incorreto:``Python was used to program''; Correto: ``we used Python''
     \end{itemize}
\end{itemize}

\section{Referências para outras fontes de informação}
%%%%%%%%%%%%%%%%%%%%%%%%%%%%%%%%%%%%%%%%

%% vim: set fo+=aw tw=80 spl=pt spell: syntax spell toplevel  :


%\chapter{Desenvolvimento}\label{chap:dese}
\chapter{Experiências e Testes}\label{chap:tests}

O referente ao ``Materials and Methods'' do capítulo~\ref{chap:back}.

Outro acrónimo pode ser \ac{TCP} (que deve estar expandido aqui, apesar de ter sido usado já no capítulo \ref{chap:devel}).

 

%\chapter{Resultados e análise}\label{chap:results}
\chapter{Resultados e análise}\label{chap:results}
 

%\chapter{Conclusões}\label{chap:conc}
\chapter{Conclusões}\label{chap:conc}



\section{Trabalho Futuro}\label{sec:trab}




%% appendix
\appendix
%% \include{app1}

%% references
%\renewcommand{\bibname}{Referências} % o babel portuguese coloca Bibliografia
% os meses do ficheiro bib poderão aparecer em inglês, caso se pretenda deve-se colocar o texto em português explicitamente no ficheiro bid
\cleardoublepage%
\phantomsection%
\addcontentsline{toc}{chapter}{\bibname}
\bibliographystyle{plainnaturlAuthor} % use plainnaturlAppear to order references by appearance 
% usually it is by author on thesis, to ease Author lookup
%\nocite{*}  % Include all entries in references.bib, not just the ones cited.
\bibliography{refs} %changed the env to make it a numbered chapter

%% bye
\end{document}
